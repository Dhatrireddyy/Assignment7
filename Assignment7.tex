\documentclass[journal,11pt,twocolumn]{IEEEtran}

\usepackage{enumitem}
\usepackage{amsmath}
\usepackage{amssymb}
\usepackage{graphicx}
\providecommand{\pr}[1]{\ensuremath{\Pr\left(#1\right)}}
\providecommand{\cbrak}[1]{\ensuremath{\left\{#1\right\}}}
\providecommand{\brak}[1]{\ensuremath{\left(#1\right)}}
\providecommand{\cdf}[2]{\ensuremath{F_{#1}\left(#2\right)}}

\title{Assignment 7}
\author{Velma Dhatri Reddy \\ \normalsize AI21BTECH11030 \\ \vspace*{10pt} \Large CBSE Probability Grade 12}
\begin{document}
\maketitle
\textbf{Exercise 13.3.3:}
Of the students in a college, it is known that 60\% reside in hostel and 40\% are day scholars (not residing in hostel). Previous year results report that 30\% of all students who reside in hostel attain A grade and 20\% of day scholars attain A grade in their annual examination. At the end of the year, one student is chosen at random from the college and he has an A grade, what is the probability that the student is a hostler?

\textbf{Solution:} 

\begin{table}[ht!]
\centering
	\input{tables/table1.tex}
	\vspace*{5pt}
\caption{}
	\label{table:table-1}
\end{table}
Event H: Student is a hostler
\begin{align}
    \pr{H} &= 60\% \\
    &= 0.6
\end{align}

Event D: Student is a day scholar
\begin{align}
    \pr{D} &= 40\% \\
    &= 0.4
\end{align}

Event A: Student gets an A grade

The probability that student gets A grade, if hostler is 
\begin{align}
    \pr{A|H} &= 30\% \\
    &= 0.3
\end{align}

The probability that student gets A grade, if hostler is 
\begin{align}
    \pr{A|D} &= 20\% \\
    &= 0.2
\end{align}

Probability that the student selected is a hostler, if he has an A grade is 
\begin{align}
    \pr{H|A} &= \frac{\pr{H}\times\pr{A|H}}{\pr{D}\times\pr{A|D}\times \pr{H}\times\pr{A|H}} \\
    &= \dfrac{0.6 \times 0.3}{0.4 \times 0.2 + 0.6 \times 0.3}\\
    &= \dfrac{0.18}{0.08 + 0.18}\\
    &= \dfrac{0.18}{0.26}\\
    &= \dfrac{0.18}{0.26}\\
    &= \dfrac{9}{13}
\end{align}

Probability that the student selected is a hostler, if he has an A grade is $\dfrac{9}{13}$.
\end{document}